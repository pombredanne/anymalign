\documentclass{beamer}
\usepackage{graphicx} % graphics
\usepackage{epsfig} % eps graphics
\usepackage{hyperref} % urls
\usepackage{booktabs, caption} % table styling
\usepackage{xcolor} % table styling
\usepackage{algorithm}
\usepackage{algpseudocode}
% code highlight, need python2-pygment
% \usepackage{minted} % gives strange output

% line spacing
\usepackage{setspace}
\setstretch{1.15}

% citation
\usepackage{biblatex}
%\usepackage{bibentry}

% set fonts
\usepackage{fontspec}
\usepackage{xunicode}

% style
\usetheme{Warsaw}
\usefonttheme{default}
\setbeamerfont{title}{size=\huge}
\setbeamerfont{frametitle}{size=\huge}

% set CJK fonts
\usepackage{xeCJK}
\setCJKmainfont{Droid Sans Fallback}

% macro
\newcommand{\inl}[1]{ \colorbox{blue!15}{$#1$} } % inline math env


\date{2013-04-05} % not \today
\title{Inside anymalign}
\author{Caesar, Crassus, Magnus }
\institute[IPS]{
  Lepage Lab \\
  Graduation school of IPS
}

\begin{document}

\section{Introduction}{
  \begin{frame}
    \titlepage
  \end{frame}

  \begin{frame}
    \frametitle{workflow of anymalign}
    \begin{algorithmic}
      \Repeat
        \begin{description}
          \item[sample] randomly pick a small subcorpora
          \item[dig]    find alignment from subcorpora
        \end{description}
      \\
      \Until
        \begin{itemize}
          \item generation rate of new alignments fall below threshold
          \item OR someone hit Ctrl-C
        \end{itemize}
      \\\ \\
      \Return
        \begin{itemize}
          \item translation probability
          \item lexical weight
        \end{itemize}
    \end{algorithmic}
  \end{frame}
}

\section{sampling}{

  \begin{frame}
    \frametitle{Size of subcorpura - Goal}
    We demand the same cover rate, from subcorpus of different size. \\
    suppose
    \begin{description}
      \item[$C$]     the cover rate
      \item[$N$]     size of corpus
      \item[$T$]     num of samples ( aka subcorpus )
      \item[$p(k)$]  probability that a corpora has size k
    \end{description}
    ideally, we have
    \begin{itemize}
      \item \inl{T \cdot p(1)} 1-sentence subcorpus covers \inl{N \cdot C} sentences
      \item \inl{T \cdot p(2)} 2-sentence subcorpus covers \inl{N \cdot C} sentences
      \item $ ... $
    \end{itemize}
  \end{frame}

  \begin{frame} \frametitle{Size of subcorpura - Modelling}
    After \inl{T \cdot p(k)} k-sentence subcorpus are randomly picked,
    \\
    the probability that a particular sentence \em{is not covered} is
    \\
    \inl{ ( (1-k/N)^k )^{ T \cdot p(k) } }
    , which equals to \inl{1-C}
  \end{frame}

  \begin{frame} \frametitle{Size of subcorpura - Conclusion}
    From \inl{((1-k/N)^k)^{ T \cdot p(k) } = 1 - C}
    \\
    we can obtain \inl{p(k) = \dfrac{ \ln{(1-C)} }{kT \ln{(1-\frac{k}{N})}} }
    \\
    OR, simply \inl{p(k) \propto \dfrac{-1}{k \ln{(1-\frac{k}{N})} } }  ( -1 keeps it positive )
  \end{frame}

}

\section{finding alignments}{

  \begin{frame}
    \frametitle{Perfect alignment}
    def:
    \\
    perfect alignment:
    \\
    word sequences that ex
    % TODO example with \tabular
  \end{frame}

  \begin{frame}
    \frametitle{Multilingual to Alingual}
    remove boundary of input languages temporialy.
    \\
    note: same literal from different languages are counted as different words
    def:
    perfect alignment:
    \\
    word sequences that appears
  \end{frame}

  \begin{frame}
    \frametitle{Grouping words}
  \end{frame}
}

\section{calculation TP,LW}{

  \begin{frame}
    \frametitle{TP}
  \end{frame}

  \begin{frame}
    \frametitle{LW}
  \end{frame}
}

\section{copy&paste}{
  \begin{frame}
    \frametitle{title}
    \begin{itemize}
      \item Hello
      \item World
    \end{itemize}
    \begin{description}
      \item[a] def-a
      \item[b] def-b
    \end{description}
  \end{frame}
}

\end{document}
