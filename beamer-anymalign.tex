\documentclass{beamer}
\usepackage{graphicx} % graphics
\usepackage{epsfig} % eps graphics
\usepackage{hyperref} % urls
\usepackage{booktabs, caption} % table styling
\usepackage{xcolor} % table styling
\usepackage{algorithm}
\usepackage{algpseudocode}
% code highlight, need python2-pygment
% \usepackage{minted} % gives strange output

% set fonts
\usepackage{fontspec}
\usepackage{xunicode}

% style
\usetheme{Warsaw}
\usefonttheme{default}
\setbeamerfont{title}{size=\huge}
\setbeamerfont{frametitle}{size=\huge}

% set CJK fonts
\usepackage{xeCJK}
\setCJKmainfont{Droid Sans Fallback}

% macro
\newcommand{\inl}[1]{ \colorbox{blue!15}{$#1$} } % inline math env


\date{2013-04-05} % not \today
\title{Inside anymalign}
\author{Caesar, Crassus, Magnus }
\institute[IPS]{
  Lepage Lab \\
  Graduation school of IPS
}

\begin{document}

\section{Introduction}{
  \begin{frame}
    \titlepage
  \end{frame}

  \begin{frame}
    \frametitle{Workflow of anymalign}
    \begin{algorithmic}
      \Repeat
        \begin{description}
          \item[sample] randomly pick a small subcorpus
          \item[find]   look for alignments from subcorpus
        \end{description}
      \Until
        \begin{description}
          \item[enough] generation rate of new alignments falls below threshold
          \item[or]
          \item[interrupted] by Ctrl-C
        \end{description}
      \Return
        \begin{itemize}
          \item translation probability
          \item lexical weight
        \end{itemize}
    \end{algorithmic}
  \end{frame}
}

\section{finding alignments}{

  \begin{frame}
    \frametitle{Perfect alignment - definition}
    \begin{description}
      \item[perfect alignment] \ \\
        words that have the same occurance vector.
    \end{description}
    \begin{itemize}
      \item scarce
        \\ \hspace{2em} in {\Large large} corpus
        \\ \hspace{2em} mostly hapaxes
      \item plentiful
        \\ \hspace{2em}in {\small small} corpus
    \end{itemize}
  \end{frame}

  \begin{frame}
    \frametitle{Perfect alignment - example}

    consider this alingual corpus: \\
    \vspace{0.5em}

    \hspace{10pt}
    \begin{tabular}{r | l }
      linenum & sentence \\
      \hline
      1 & we are cool .
      \\
      2 & are not we cool ?
      \\
      3 & agreed , we are surely cool .
      \\
    \end{tabular}
    % we     <1,1,1>
    % are    <1,1,1>
    % cool   <1,1,1>

    % .      <1,0,1>

    % not    <0,1,0>
    % ?      <0,1,0>

    % ,      <0,0,1>
    % argeed <0,0,1>
    % really <0,0,1>

    \vspace{2em}
    we classify words by their occurance vector: \\

    \vspace{0.5em}
    \hspace{10pt}
    \begin{tabular}{r | l }
      vector & set of words \\
      \hline
      $<1,1,1>$ & \{ \inl{we}, \inl{are}, \inl{cool} \}
      \\
      $<0,0,1>$ & \{ \inl{,} , \inl{agreed}, \inl{surely} \}
      \\
      $<1,0,1>$ & \{ \inl{.} \}
      \\
      $<0,1,0>$ & \{ \inl{?}, \inl{not} \}
      \\
    \end{tabular}
  \end{frame}

  \begin{frame}
    \frametitle{Perfect alignment - what are they ? }

    \begin{description}
      \item[in alingual corpus] \ \\
        `just some particular case of collocations` \\
        --- \fullcite{lardilleux-lepage:2009:RANLP09}
        \vspace{1em}
      \item[in multilingual corpus] \ \\
        still collocations of course, \\
        and likely to be  phrases that translates to each other
    \end{description}

  \end{frame}

  \begin{frame}
    \frametitle{Perfect alignment - align as alingual}

    To find translation candidates using perfect alignments
    \begin{description}
      \item[merge]
        concatenate aligned sentences into one \\
        { \small ※ a.k.a `remove boundary of input languages` } \\
        { \small ※ same literal from different languages are still counted as different words }
      \item[find] prefect alignments from the looooooooong sentence,
        and yield
        \begin{itemize}
          \item perfect alignments
          \item complementary of perfect alignments
        \end{itemize}
        as aligned words.
          % - perfect alignments themselves, \\
          % - bb \\
        %rest of 
        % perfect alignments,
        % as if they were from an alingual sentence
      \item[split]
        put boundary of languages back into aligned words,
        and (hopefully) get aligned phrases.
    \end{description}
  \end{frame}

  \begin{frame}
    \frametitle{Perfect alignment - give me more }
    reminder:
    \begin{itemize}
      \item scarce in {\Large large} corpus
      \item plentiful in {\small small} corpus
    \end{itemize}
    \vspace{1em}
    \inl{\Rightarrow} pick subcorpura
    \\ \hspace{2em} while biasing toward smaller ones
  \end{frame}

}

\section{sampling}{

  \begin{frame}
    \frametitle{Sampling - Modelling}
    We assume the same cover rate, from subcorpura of different size. \\
    let
    \begin{description}
      \item[$C$]     the cover rate
      \item[$N$]     size of corpus
      \item[$T$]     num of samples ( a.k.a subcorpura )
      \item[$p(k)$]  probability that a corpus has size k
    \end{description}
    ideally, we have
    \begin{itemize}
      \item \inl{T \cdot p(1)} 1-sentence subcorpura covers \inl{N \cdot C} sentences
      \item \inl{T \cdot p(2)} 2-sentence subcorpura covers \inl{N \cdot C} sentences
      \item $ ... $
    \end{itemize}
  \end{frame}

  \begin{frame} \frametitle{Sampling - Modelling}
    After \inl{T \cdot p(k)} k-sentence subcorpus are randomly picked,
    \\ \  \\
    the probability that a particular sentence {\bf is not covered} is
    \\
    \inl{ ( (1-k/N)^k )^{ T \cdot p(k) } }
    , which equals to \inl{1-C}
  \end{frame}

  \begin{frame} \frametitle{Sampling - Conclusion}
    From \inl{((1-k/N)^k)^{ T \cdot p(k) } = 1 - C}
    \\
    we can obtain \inl{p(k) = \dfrac{ \ln{(1-C)} }{kT \ln{(1 - k/N)}} }
    \\
    OR, simply \inl{p(k) \propto \dfrac{-1}{k \ln{(1-k/N)} } }  ( -1 keeps it positive ),
    \\
    \vspace{2em}
    which fits our needs well.
  \end{frame}

}

\section{yield}{

  \begin{frame}
    \frametitle{Translation probability}
    As its multilingual nature, TPs are given in {\em one-to-many} manner. \\
    Each language becomes {\em source} in turn, all the rest being {\em target}.

    \vspace{0.5em}
    If
    \begin{tabular}{ r | r | r | l }
      en      & zh   & ja   & count \\
      \hline
      table & 桌子 & テーブル & 2 \\
      table & 表格 & 表 & 1 \\
    \end{tabular}
    are the alignments we found, \\
    \vspace{1em}
    corresonding TP can be calculated by \\
    \begin{tabular}{ r | r | l }
      target & source & translation probability \\
      \hline
      \small{桌子,テーブル} & table & $ \frac{2}{1+2}  $ \\
      \small{表格,表} & table & $ \frac{1}{1+2}  $ \\
      \small{table,桌子} & テーブル & $ \frac{2}{2}  $ \\
      \small{table,表格} & 表 & $ \frac{1}{1}  $ \\
      ... & & \\
    \end{tabular}
  \end{frame}

  \begin{frame}
    \frametitle{Lexical weight}
    Similarily, LWs are computed in {\em one-to-many} manner. \\

    \ \\
    As anymalign did not give word-to-word TPs by far,
    we have to estimate them by counting: \inl{ D(word_i|word_j) = \frac{count(word_i,word_j)}{count(word_j)} } \\

    \ \\
    Then, lexical weight of
    \inl{ t_1, t_2, ... \leftarrow s }
    are product of
    \inl{ D(\texttt{word in t} | \texttt{word in s} ) }
    that maximizes the weight, namely \\
    \ \\
    \inl{
      LW( t_1, t_2 ... | s ) = \prod_{w_j \in s}{\max_{w_i \in t }D(w_i|w_j) }
    }
  \end{frame}
}

\section{copy&paste}{
\begin{frame}
  \frametitle{it's over}
  thank you
\end{frame}
}

\end{document}
